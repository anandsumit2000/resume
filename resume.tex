%%%%%%%  STARTING HERE, Document Formatting Settings


\documentclass[a4,10pt]{article}
\usepackage{latexsym}
\usepackage[empty]{fullpage}
\usepackage{titlesec}
 \usepackage{marvosym}
\usepackage[usenames,dvipsnames]{color}
\usepackage{verbatim}
\usepackage[hidelinks]{hyperref}
\usepackage{fancyhdr}
\usepackage{multicol}
\usepackage{hyperref}
\usepackage{csquotes}
\usepackage{tabularx}
\hypersetup{colorlinks=true,urlcolor=black}
\usepackage[11pt]{moresize}
\usepackage{setspace}
\usepackage{fontspec}
\usepackage[inline]{enumitem}
\usepackage{array}
\newcolumntype{P}[1]{>{\centering\arraybackslash}p{#1}}
\usepackage{anyfontsize}
\usepackage[margin=1cm, top=1cm]{geometry}
\setmainfont[
BoldFont=SourceSansPro-Semibold.otf, %SourceSansPro-Bold.otf
ItalicFont=SourceSansPro-RegularIt.otf
]{SourceSansPro-Regular.otf}
\setsansfont{SourceSansPro-Semibold.otf}
\pagestyle{fancy}
\fancyhf{} 
\fancyfoot{}
\renewcommand{\headrulewidth}{0pt}
\renewcommand{\footrulewidth}{0pt}
\urlstyle{same}
\raggedbottom
\raggedright
\setlength{\tabcolsep}{0in}
\definecolor{UI_blue}{RGB}{32, 64, 151}
\usepackage[style=nature, maxbibnames=3]{biblatex}
\addbibresource{Publications.bib}
\titleformat{\section}{
\color{UI_blue} \scshape \raggedright \large 
}{}{0em}{}[\vspace{-10pt} \hrulefill \vspace{-6pt}]
\newcommand{\subtext}[1]{
#1\par\vspace{-0.2cm}}
\setlist[itemize]{align=parleft,left=0pt..1em}
\newenvironment{zitemize}{
\begin{itemize}\itemsep0pt \parskip0pt \parsep1pt}
{\end{itemize}\vspace{-0.5cm}}
\newcommand{\hskills}[1]{
\textbf{\bfseries #1} }
\titleformat{\subsection}{\vspace{-0.1cm} 
\bfseries \fontsize{11pt}{2cm}}{}{0em}{}[\vspace{-0.2cm}]

%%%%%%%  END OF "Document Formatting Settings" SECTION


\begin{document}

\begin{center}
    \begin{minipage}[b]{0.24\textwidth}
            \ (+91)-9156011787 \\
            \normalsize \href{mailto:as.anandsumit@gmail.com}{as.anandsumit@gmail.com} 
    \end{minipage}% 
    \begin{minipage}[b]{0.5\textwidth}
            \centering
            {\Huge \underline {SUMIT ANAND}} \\ %
            \vspace{0.1cm}
            {\color{UI_blue} \Large{DevOps and Site Reliability Engineer}} \\
    \end{minipage}% 
    \begin{minipage}[b]{0.24\textwidth}
            \flushright \normalsize  %Willing to Relocate
            {\href{https://www.linkedin.com/in/asanandsumit/}{linkedin/asanandsumit} } \\
            \href{https://github.com/anandsumit2000}{GitHub/anandsumit2000}
    \end{minipage}   

    {\color{UI_blue} \hrulefill}
\end{center}

Site Reliability Engineer with 2 years of experience adept at developing robust monitoring tools, disaster recovery, and automation solutions. Catalyzing efficiency by transforming operations with strategic automations and ensuring high availability of systems.  %% This can be customized, but I feel this impact statement makes people want to read more.

\section{Education }

\subsection*{B.E, Computer Science Engineering} 
{{\normalsize \normalfont Cambridge Institute Of Technology - Bangalore, KA} \hfill Aug 2018 --- July 2022} 

\section{Experience}

%%%%%%% ----------------------------------- Role 5 ----------------------------------- %%%%%%%
\subsection*{Associate Software Engineer - DevOps \hfill Aug 2022 --- Present} 
\subtext{Syncron Software India Pvt. Ltd \hfill Bangalore, KA} 
    \begin{zitemize}
        \item Architected a new solution with a custom authentication layer using AWS Lambda and API Gateway on AWS Transfer Family, replacing the legacy open-source SFTP system at the service.
        \item Enhancing the CI/CD deployment method that utilizes the CDK Toolkit to empower developers to dynamically deploy changes to the data-platform(AWS Data.All) via GitHub webhooks and AWS CodePipeline.
        \item Implemented GitHub workflows to regularly scan repositories for deprecated packages, addressing vulnerabilities and potential breaches. Successfully reduced associated risks by 30\%.
        \item Upgraded EKS cluster back-ends with persistent type FSx storage volumes for resiliency and fault-tolerance. Wrote shell scripts for housekeeping the clusters for maintenance and recovery.
        \item Produced terraform scripts for provisioning resources to securely onboard businesses to the dynamic data movement platform service. Brought in automation to fetch and rotate logs from ECS fargate clusters for fault analysis. 
        \item Implemented disaster recovery plan, minimizing downtime by 90\% to enhance system resilience, guaranteeing uninterrupted service for 500+ business managers allowing them to forecast and strategize for maximum business growth. 
    \end{zitemize}


%%%%%%% ----------------------------------- Role 4 ----------------------------------- %%%%%%%
\subsection*{DevOps Engineer {\normalsize\normalfont (Intern)} \hfill Apr 2022 --- July 2022} 
\subtext{Syncron Software India Pvt. Ltd \hfill Bangalore, KA} 
    \begin{zitemize}
        \item Launched Lambda function to push Redshift Database metrics to CloudWatch, reducing issue response time by 45\%.
        \item Assisted upgrades for legacy Terraform repositories, achieving a 40\% reduction in deployment time.
    \end{zitemize}
